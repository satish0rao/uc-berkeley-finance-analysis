\documentclass[hyperref={colorlinks}]{beamer}


\usepackage{times,latexsym,mathptmx,comment,latexsym,epsf,amssymb,graphicx,url,amsmath,amsfonts,textcomp}

\newcommand{\skippause}{\vskip5pt
\pause}

\begin{document}

\small
\begin{frame}{My position.}

Bored with the obvious! \\ 
~~The administration is terrible and \\
~~consumes massively too much money.
\skippause
My position: we need to formulate a strategy to fix this.
\skippause
Ok, ok: the obvious...
\skippause
~~Briefly......

\end{frame}

\begin{frame}{Absolute Value.}

From \href{http://reportingtransparency.universityofcalifornia.edu/}{Berkeley Expense Numbers.} \\

Berkeley: \\

Cost of Instruction: {\color{red} 855 Million.} \\
\skippause
Total (State) Salary Cost of Faculty/Teachers/GSIs: {\color{blue} 270 M.} \\
Benefits: {\color{blue} $\leq$ 130M (including tuition?)} \\
So..{\color{blue} 400M} \\ 
\skippause
Where oh, where did the {\color{red} 455M} go? \\
\skippause
Student Services? No! \\
~~~Separate Line Item: {\color{green} 202M}\\
~~~~Total Direct Student Service Employee Pay: {\color{blue} 56M} (including physicians.)\\
\skippause
Academic Support? No! \\ 
~~~Separate Line Item: {\color{green} 165M} \\
\skippause
Institutional Support? No! \\
~~~Separate Line Item: {\color{green} 186M}  \\
\skippause
Where oh, where did the {\color{red} 455M} go? \\
{\tiny \href{https://github.com/satish0rao/uc-berkeley-finance-analysis}{Salary Scripts Github.}}
\end{frame}

\begin{frame}{Growth...}

Sanjay: Ask  him! \\
~~~{\color{blue} Can administration cuts alone address all or the bulk of deficit?}
\skippause
Schwartz (Mar 29, 2016): \\
~~~new data from California Public Records Act \\
~~~~~~~~~(Note: I can analyse if someone proposes how.) \\ 
~~~Management/finance consume ever increasing fraction.  \\
~~~Berkeley appears to be the worst of campuses.\\
~~~~~~(Health Sciences is worse.) \\
\skippause
Rao: notes... \\ 
~~~Finance/MGR versus Faculty: 2012-2014. \\
~~~~Finance/MGR 26\% growth in salary cost. \\
~~~~Faculty  9\% growth in salary costs. \\
~~~Affiliate/Non-Employees: {\color{red} 1774 (2008) to 4202 (2016)} \\
~~~~~~~~~~Who are these people? Why the growth? \\
~~~{\color{red} 100 M} in other provisions added to Cost of Instruction in 2015. \\
\end{frame}


\begin{frame}{Can another do better?}

Santa Barbara is most comparable non-medical school campus.\\
\skippause
~~Total salary of researchers/teachers/gsi/...
\skippause
Direct research/teacher/gsi percentage of total Berkeley salary: \\
~~Berkeley: 37.5\%  \\
~~Santa Barbara: 45\% \\
\skippause
We spend roughly {\color{red} 20\%} more on ``Other'' Salary. \\
~~This totals to 120M in salary. \\
~~~add benefits we get to {\color{blue} 168M} excess spending.\\
\skippause
{\color{red} Claim(needs vetting): \\
 Can fix budget with UCSB level efficiency on non-teaching/research employee cost {\color{blue} alone.}}
\skippause
Notes: \\
~~~Needs vetting!  Questions? \\
~~~I don't think Santa Barbara is well run either: \\
~~~~~follows the same corporate model of taking from students \\
~~~~~noted in slide on absolutes. 
\skippause


\end{frame}

\begin{frame}{Doing well? How do we tell?}

Faculty: publications, grants, citations, influence, 
{\color{blue} student evaluation of teaching}, reputation. 
\skippause
Students: SAT, GPA, selectivity, graduation, salary, placement..
\skippause
Administration: {\color{red} ?, ? , ? , ? ...}
\skippause
~~~One numerical example: \\
~~~~Princeton Review ranking of administrations. \\
~~~~{\color{red} Bottom 20 out of 384 colleges: 380K student surveys.}
\skippause
~~~Consistent with my experience both personal and  \\
~~~~from colleages/visitors/students...

\end{frame}

%% \begin{frame}{Prof Charles Schwartz}


%% \skippause
%% Caveat: wants research part of faculty separated out with respect
%% to undergraduate teaching?

%% \end{frame}

\begin{frame}{Random Concluding Thoughts.}

We spend too much money on people
not providing direct services to students.
\skippause
The corporate model specifically minimizes cost of labor. 
\skippause
Monopolistic position of selective universities (85K applicants, 17\%
admission rate) allows growth of employees whose value is hard to
measure.
\skippause
UC can't charge monopolistic prices due to political environment.
\skippause
If not ok, is there a strategy to change?
\skippause
What do people think?  Send questions.


\end{frame}


\begin{frame}{Other slides: todo.}

\end{frame}

\begin{frame}{Assembly Bill 94}

\href{http://www.ucop.edu/operating-budget/_files/legreports/14-15/efifinallegrpt-2-17-15.pdf}{Pretty uninformative final report}

\end{frame}

\begin{frame}{Mechanics around calculations..}

\skippause
From Charlie Schwartz: \\
~~~BACUBO: National Association of College and University Business Officers \\
\skippause
~~~~Cost of Instruction: differs college to college. \\ 
~~~~
~~~~Cost of Education- contains \pause Museums, \pause, libraries, \pause, housing/dining if deficit, \pause, recreation, \pause intercollegiate athletics deficits, \pause financial aid (double counting?) \pause ... \\
\skippause
NACUBO report\ref{http://www.nacubo.org/documents/research/cofcfinalreport.pdf}.

\end{frame}

\end{document}


