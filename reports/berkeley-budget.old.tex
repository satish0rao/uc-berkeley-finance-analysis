\documentclass{article}

\title{{\bf WORKING DRAFT:}\\  Overhead at UC Berkeley.}
\author{}

\newcommand{\pt}{{\cal{P}_{\cal{T}}}}
\newcommand{\ptnull}{{\cal{P}_{\cal{T_{\perp}}}}}
\newcommand{\sampleq}{{\tilde{\cal R^q}}_{\Omega}}
\newcommand{\sampleqk}{\tilde{{\cal R^q}}_{\Omega_k}}
\newcommand{\samplesk}{{\cal R}_{\Omega_l}}
\newcommand{\samples}{{\cal R}_{\Omega}}


\begin{document}
\maketitle

\begin{abstract}

We compare the effective overhead at UB Berkeley against UC Santa Barbara,
by identifying the part of the budget that goes towards direct support of teaching
and research versus administrative costs. The effective overhead of this calculation 
for UC Berkeley is roughly 235\%, while for UC Santa Barbara it is roughly 159\%. While neither 
figure is small, it does suggest that bringing UC Berkeley's numbers more in line 
with UC Santa Barbara's would generate significant savings, to the extent of 
completely dealing with the budget deficit.


\end{abstract}

\section{Introduction}

We discuss Berkeley's overheads on teaching and research largely by
comparison with UC Santa Barbara, the next largest member of the UC
System that does not have an associated medical center. This ongoing
analysis centers on salaries which along with associated benefits
comprise the bulk of Berkeley's expenditures. The analysis suggests
that Berkeley could save in excess of $100 million annually, with our 
estimates suggesting numbers more on the order of $200 million dollars,
if its overheads were more in line with those at UC Santa Barbara. 

We make no claim nor do we believe UC Santa Barbara especially
effective, just more effective than Berkeley. 

We note this is a working document, and surely includes errors and 
misinformation which we would welcome feedback on. 

We note there is perhaps something to learn from best practices
elsewhere beyond employee compensation. For example, it appears
Santa Barbara escapes significant expenses (on the order of ...) in student healthcare
which Berkeley does not (and to be fair, neither does UCLA).See
section~\ref{sec:healthcare}. A particularly area where Berkeley overhead precentage 
significantly exceed UCSB's is management expenditures --- 136M (59\%) versus 29M (26\%).

\section{Overhead Calculations}

\subsection{IPEDS 2014.}

From 2014 ipeds datafiles \cite{ipeds-is}, we have that UC
Berkeley spent 232 million dollars on faculty salary to UC Santa
Barbara's 112 million.  For non instructional salary totals,
we have 653 million versus 208 million \cite{ipeds-nis}. 

The non-instructional salaries include research salaries which are
respectively 82 million and 25 million, and Berkeley also spends
significantly more, 25 million versus 4 million, on the category of
``Librarians, Curators, Archivists and Academic Affairs and Other
Education Services - outlays'' which we also take to directly serve
our mission.  

Deducting those costs leaves 546 million versus 179 million for 
UC Santa Barbara. 

{\bf Basic Calculation.} The effective overhead of this calculation for UC Berkeley is 235\%,
and 159\% for UC Santa Barbara.  Were Berkeley consistent with Santa
Barbara on this overhead, its noninstructional salaries would engender
a savings of 177 million in salary alone: the 546 million of overhead would
be reduced to $(1.59 \times 232) = 369$ million. 

Using an estimated associated benefit rate of 36\% as suggested by
\cite{rev-trends},  yields a saving of 241 million dollars in
salaries and benefits alone using data from 2014 data.  More recent
data which we discuss below suggests a widening gap between Berkeley
and Santa Barbara.

Alternatively, one could argue that research induces overhead
as well.  To be sure, it should be on the order of 57\%. Thus,
we adjust this calculation in two ways. 

{\bf Large Research Overhead Calculation.} We add research and faculty
salaries which yield totals 314 million and 137 million
respectively. The associated overhead rates are then respectively
174\% and 152\%.  The associated savings for Berkeley were these equal
94 million (using the calculation $1.36\times (546 - 314\times1.52)$).

{\bf Modest Research Overhead Calculation.} On the other hand, were
the research overheads limited to say 57\% (our typical research
overhead rate), we remove that from the overheads instead getting
``teaching overheads'' of 499 million ($546 - .57\times 82$) and 165
million, respectively.  This teaching overhead rate then becomes 215\%
and 147\%.  Equalizing that overhead would save 190 million.

\subsubsection{Some places to look.}

These are from the IPEDS non-instructional staff data
\cite{ipeds-nis}.  The base for computing overhead percentages, we
will use is faculty salaries (232M or 112M).

{\bf Management Overheads.} 136M (59\%) versus 29M (26\%). Imputed
savings: 75M in salary, 102M total compensation.

{\bf Business/Finance:} 97M ( 42\%) versus 31M (28 \%). Imputed savings:
32M in salary, 43M total compensation.

{\bf Student Services:} 64M (28\%) versus 19M (17\%). Imputed savings:
25M in salary,  34 M in total compensation.

\subsection{Salary data}

We also worked with salary data files \cite{salary,ucpay}.
Here we heuristically computed direct and indirect
salary expenditures for 2015. The computation is embodied in a
publicly available repository \cite{github-link}.


{\bf Regular Pay.}  Our heuristics estimate total direct salaries
(teaching,research, museum, agronomists) as 379 million and 173
million out of total salaries 1082 million and 415 million for
Berkeley and Santa Barbara respectively.  This is using the regular
pay fields in that data.

Imputed overhead rates are 185\% and 140\%.  Equalizing
would save Berkeley 170M in salary or 232M in total
compensation. 

{\bf Total Wages.} Using total wages rather than base pay
(e.g., including summer salary), we get 440M and 194M
out of 1185M and 447M.  Imputed overhead rates are 

Imputed overhead rates are 169\% and 127\%.  Equalizing
would save Berkeley 185M in salary or 252M in total
compensation. 

{\bf Adjusting for auxillary.} One could legitimately
remove things like housing associated salaries from overhead since
they are paid for from pools that are distinct
from the research/teaching/public service missions.
Thus, we use data from 2015 Consolidated Financial
Reports \cite{UCSB-consolidated,UCB-consolidated}
which suggest that auxilary salaries and wages
are 46M and 36M respectively. (One should be very
impressed as Santa Barbara has four year housing
I believe.)

With these adjustments to say the total wage
scenario we get imputed overhead rates of
158\% and 112\%.  Equalizing would save
Berkley 180M in salary or 245M in total
compensation.

\section{Other Questions/Issues.}

\begin{enumerate}
\item
{\bf Student health expenditures} from the Berkeley Consolidated
Report \cite{UCB-consolidated} in 2015 has a non-salary expense of
72M.  For Santa Barbara \cite{UCSB-consolidated} this is
11M. Admittedly, UCLA \cite{UCLA-consolidated}
also suffers a large non-salary expense for this category.

\item
{\bf The number of Affiliates/Non-Employees} has moved from 1921
in 2008 to 4209 in 2016 \cite{cal-answer-census}.  Do these people incur expense? Are
they included in salary data?

\item
{\bf Other Provisions} increased from a XXX in the 2014 Consolidated Financial 
Statement to XXX in the 2015 Consolidated Financial Statement.  This amount
is assigned to Cost of Instruction (as for example is the vast majority
of the Deans offices).   One wonders what Other Provisions is.  The 
amounts for UC Santa Barbara and UCLA are respectively XXX and XXX. 

\item
TODO: Compute a direct teaching overhead on all tuition/state supported
activities. 

\end{enumerate}

\section{Some remarks.}

Some suggest that we need to admit and register students. According
to the Berkeley Consolidated Statement, these salary expenses
are 4M and 3.4M respectively.

For comparison purposes, Dean's offices total expenses are around
47.5M and Vice Chancellors another 34M.  Dean's salaries alone
cost 10M. 

Incidentally, faculty brought in roughly 748M in research funds in
2015, and the overhead on this of 175M.  Moreover, millions more on
graduate student researcher tuition goes directly to the university
for roughly 2000 positions.  For comparison purposes faculty
(non-summer,non-lecturer) wages are around 210M.



\bibliographystyle{plain}
\bibliography{reports}


\end{document}
