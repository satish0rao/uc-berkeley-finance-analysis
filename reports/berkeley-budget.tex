\documentclass{article}

\title{{\bf WORKING DRAFT:}\\  Overhead at UC Berkeley.}
\author{}

\newcommand{\pt}{{\cal{P}_{\cal{T}}}}
\newcommand{\ptnull}{{\cal{P}_{\cal{T_{\perp}}}}}
\newcommand{\sampleq}{{\tilde{\cal R^q}}_{\Omega}}
\newcommand{\sampleqk}{\tilde{{\cal R^q}}_{\Omega_k}}
\newcommand{\samplesk}{{\cal R}_{\Omega_l}}
\newcommand{\samples}{{\cal R}_{\Omega}}


\begin{document}
\maketitle

\begin{abstract}

We compare the effective overhead at UC Berkeley against UC Santa
Barbara, by identifying the part of the budget that goes towards
directly to teaching and research versus overhead costs. The effective
overhead of one such calculation for UC Berkeley is roughly 215\%, while
for UC Santa Barbara it is roughly 159\%. While neither figure is
small, it does suggest that bringing UC Berkeley's numbers more in
line with UC Santa Barbara's would generate significant savings, to
the extent of completely dealing with the budget deficit.

We also present alternative calculations and welcome guidance.

\end{abstract}

\section{Introduction}

We discuss Berkeley's overheads on teaching and research largely by
comparison with UC Santa Barbara, the next largest member of the UC
System that does not have an associated medical center. This ongoing
analysis centers on salaries which along with associated benefits
comprise the bulk of Berkeley's expenditures. These analyses suggests
that Berkeley could save in excess of \$100 million annually, with our
best estimates suggesting numbers more on the order of \$200 million
dollars, if UC Berkeley overheads were in line with those at UC
Santa Barbara.

We make no claim nor do we believe UC Santa Barbara especially
effective, just more effective than UC Berkeley. 

We note this is a working document, and highly value feedback.

We note there is perhaps something to learn from best practices
elsewhere beyond employee compensation. For example, it appears Santa
Barbara escapes significant expenses (on the order of \$77 million) in student
healthcare which Berkeley does not (and to be fair, neither does
UCLA).See section~\ref{sec:healthcare}. A particularly area where
Berkeley overhead precentage significantly exceed UCSB's is management
expenditures --- \$136 million (59\% overhead rate)
versus \$29 million (26\% overhead rate).

\section{Overhead Calculations}

\subsection{IPEDS 2014.}

From 2014 ipeds datafiles \cite{ipeds-is}, we have UC
Berkeley spending \$232 million on faculty salary to UC Santa
Barbara's \$112 million.  For non instructional salary totals,
we have \$653 million versus \$208 million \cite{ipeds-nis}. 

The non-instructional salaries include research salaries which are
respectively \$82 million and \$25 million, and Berkeley also spends
significantly more, \$25 million versus \$4 million, on the category
of ``Librarians, Curators, Archivists and Academic Affairs and Other
Education Services - outlays'' which we also take to directly serve
our research, teaching and public service mission.

Deducting those costs leaves 546 million versus 179 million for UC
Santa Barbara for total overhead.

{\bf Basic Calculation.} The effective overhead of this calculation
for UC Berkeley is 235\%, and 159\% for UC Santa Barbara.  Were
Berkeley consistent with Santa Barbara on this overhead, its
noninstructional salaries would engender a savings of 177 million in
salary alone: the 546 million of overhead would be reduced to $(1.59
\times 232) = 369$ million.

Using an estimated associated benefit rate of 36\% as suggested by
\cite{rev-trends},  yields a saving of 241 million dollars in
salaries and benefits alone using data from 2014 data.  More recent
data which we discuss below suggests a widening gap between Berkeley
and Santa Barbara.

Alternatively, one could argue that research induces overhead as well.
The first yields a possibly illegal overhead rate on federal funds of
174\% for Berkeley.  Thus, we include a calculation where the rate is
limited to 57\%: the federally negotiated overhead on research grants.

{\bf Large Research Overhead Calculation.} We add research and faculty
salaries which yield totals 314 million and 137 million
respectively. The associated overhead rates are then respectively
174\% and 152\%.  The associated savings for Berkeley were these equal
94 million (using the calculation $1.36\times (546 - 314\times1.52)$).

{\bf Standard Research Overhead Calculation.} On the other hand, were
the research overheads limited to say 57\% (our typical research
overhead rate), we remove that from the overheads instead getting
``teaching overheads'' of 499 million \footnote{We calculate this as follows: $546 - .57\times 82$, reducing the \$546 million by the federal
overhead on \$ 82 million.} and 165
million, respectively.  This teaching overhead rate then becomes 215\%
and 147\%.  Equalizing that overhead would save 190 million.

\subsubsection{Specific areas for further examination.}

These are from the IPEDS non-instructional staff data
\cite{ipeds-nis}.  The base for computing overhead percentages, we
will use is faculty salaries (\$232M or \$112M).

{\bf Management Overheads.} \$136M (59\%) versus \$29M (26\%). Imputed
savings: \$75M in salary, \$102M total compensation.

{\bf Business/Finance.} \$97M ( 42\%) versus \$31M (28 \%). Imputed savings:
\$32M in salary, \$43M in total compensation.

{\bf Office and Administrative Support.}  \$69M (30\%) versus
\$20M (18\%). Imputed savings: \$28M in salary, \$38M in total compensation.

{\bf Student Services.} \$64M (28\%) versus \$19M (17\%). Imputed savings:
\$25M in salary, \$34 M in total compensation. 

{\bf Computer, Engineering, and Science.} \$105M (45\%) versus \$37M
(33\%). Imputed savings: \$28M in salary, \$38M in total compensation.
\footnote{We expect this particular category may have good justification via differences in research activity.}


\subsection{Salary data}

We also worked with salary data files \cite{salary}.
Here we heuristically computed direct and indirect
salary expenditures for 2015. The computation is embodied in publicly
available code \cite{github-link}.

{\bf Regular Pay.}  Our heuristics estimate total direct salaries
(teaching,research, museum, agronomists) as \$379 million and \$173
million out of total salaries \$1082 million and \$415 million for
Berkeley and Santa Barbara respectively.  This is using the regular
pay fields in that data.

Imputed overhead rates are 185\% and 140\%.  Equalizing
would save Berkeley \$170M in salary or \$232M in total
compensation. 

{\bf Total Wages.} Using total wages rather than base pay
(e.g., including summer salary), we get \$440M and \$194M
out of \$1185M and \$447M.  Imputed overhead rates are 

Imputed overhead rates are 169\% and 127\%.  Equalizing
would save Berkeley \$185M in salary or \$252M in total
compensation. 

{\bf Adjusting for auxillary.} One could legitimately
remove things like housing associated salaries from overhead since
they are paid for from pools that are distinct
from the research/teaching/public service missions.
Thus, we use data from 2015 Consolidated Financial
Reports \cite{UCSB-consolidated,UCB-consolidated}
which suggest that auxilary salaries and wages
are \$46M and \$36M respectively. We note that.

With these adjustments to say the total wage
scenario we get imputed overhead rates of
158\% and 112\%.  Equalizing would save
Berkley 180M in salary or 245M in total
compensation.

\section{Other Questions/Issues.}
\label{sec:healthcare}

\begin{enumerate}
\item
{\bf Student health expenditures} from the Berkeley Consolidated
Report \cite{UCB-consolidated} in 2015 has a non-salary expense of
\$72M.  For Santa Barbara \cite{UCSB-consolidated} this is
\$11M. Admittedly, UCLA Consolidated Report \cite{UCLA-consolidated}
also contains a large non-salary expense for this category.

\item
{\bf The number of Affiliates/Non-Employees} has moved from 1921 in
2008 to 4209 in 2016 \cite{cal-answer-census}.  Do these people incur
expense? Are they included in salary data?

\item
{\bf Other Provisions} increased from a \$20M in the 2014 Consolidated
Financial Statement \cite{UCB-consolidated-14} to \$123M in the 2015
Consolidated Financial Statement \cite{UCB-consolidated}.  This amount
is assigned to Cost of Instruction (as for example is the vast
majority of the Deans offices).  One wonders what Other Provisions is,
especially with respect to instruction.  The amounts for UC Santa
Barbara and UCLA are non-existent with respect to instruction.

\item
{\bf Financial Aid}. Santa Barbara appears to provide more
percentage-wise in terms of financial aid: \$135M(58\%)
for Berkeley versus \$83M (74\%).  

\end{enumerate}

\section{Some remarks.}

For Berkeley to function, one certainly needs many people beyond
teachers and researchers to properly accomplish our mission. For
example, we certainly need an admissions office and a registrar. These
absolutely central tasks, however, cost a modest \$4.5M and \$3.5M in total.
Berkeley, moreover, is slightly more efficient than Santa Barbara in this
categories.

For comparison purposes, Dean's offices total expenses are around
\$47.5M and Vice Chancellors another \$34M.  Again, Santa Barbara does
better in these categories.

Incidentally, faculty brought in roughly \$748M in research funds in
2015, and the overhead on this of \$175M.  Moreover, millions more on
graduate student researcher tuition goes directly to the university
for roughly 2000 GSR positions.  For comparison purposes faculty
(non-summer,non-lecturer) wages are around \$210M.

\section{Data Remarks.}

The Integrated Postsecondary Education Data System is maintained by
the National Center for Education System and collects information from
colleges and university using standardized methodologies.

\bibliographystyle{plain}
\bibliography{reports}


\end{document}
