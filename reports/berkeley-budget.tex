\documentclass[11pt]{article}

\title{{\bf WORKING DRAFT:}\\  Overhead at UC Berkeley.}
\author{}

\newcommand{\pt}{{\cal{P}_{\cal{T}}}}
\newcommand{\ptnull}{{\cal{P}_{\cal{T_{\perp}}}}}
\newcommand{\sampleq}{{\tilde{\cal R^q}}_{\Omega}}
\newcommand{\sampleqk}{\tilde{{\cal R^q}}_{\Omega_k}}
\newcommand{\samplesk}{{\cal R}_{\Omega_l}}
\newcommand{\samples}{{\cal R}_{\Omega}}

\addtolength{\oddsidemargin}{-.5in}
\addtolength{\evensidemargin}{-.5in}
\addtolength{\textwidth}{1.2in}

\addtolength{\topmargin}{-.875in}
\addtolength{\textheight}{1.2in}


\begin{document}
\maketitle

\begin{abstract}

We conduct some analyses that, put simply, suggests that, scaling for size, whereas UC
Berkeley should be spending roughly twice as much as UC Santa Barbara
on non-teaching/non-research salaries, instead it spends in excess of two and
a half times as much.  A move to UCSB's level of
``overheads'' would save hundreds of millions of dollars in
compensation alone.  The bulk of these savings appear in Management,
Administration, and Finance areas and some savings also appear in
Computing and Student Services.

In particular, we compute an ``effective overhead'' which is the ratio
of non-instructional/non-research salary to the salary of direct
instructional and research personnel. This measure is a convenient
rough measure of efficiency of the institutions in that equal numbers
roughly imply equal efficiency. 

The calculations indicate that the effective overhead rate for UC
Berkeley is roughly 215\%, while for UC Santa Barbara it is roughly
159\%. While neither figure is small, it does suggest that bringing UC
Berkeley's numbers more in line with UC Santa Barbara's would generate
significant savings, quite possibly, in excess of the budget deficit
of \$150 million.

We welcome feedback about any potential gaps in this analysis.

\end{abstract}

\section{Introduction}

We apply a simple measure of institutional efficiency of our universities,
which we call {\bf effective overhead rate}: it is the ratio of salary
spent on non-direct teaching and research efforts to the salary spent
on those directly teaching and doing research. Computing this simple rate can 
be very revealing, since we expect that equally
efficient universities should have comparable effective overhead rates.
We note that the effective overhead rate is somewhat analogous to 
the research overhead rate used for federal and other grants.


We compare UC Berkeley's effective overhead rate to that of UC Santa
Barbara, the next largest member of the UC System that does not have
an associated medical center.  Our analysis suggests that a move to
UCSB's level of effective overhead would save hundreds of millions of
dollars in compensation alone.  The bulk of these potential savings
appear in Management, Administration and Finance function and some
savings also appear in Computing and Student Services.

Specifically, the effective overhead rate for UC Berkeley is roughly
215\%, while for UC Santa Barbara it is roughly 159\%.  In particular,
an area where UCB's overhead percentage significantly exceeds UCSB's
is management expenditures --- \$136 million (59\% overhead rate)
versus \$29 million (26\% overhead rate).  We make no claim, nor do we
believe that UC Santa Barbara is especially efficient with respect to
overhead rate --- it just appears to be more efficient than UC
Berkeley.

%% As noted above, our overhead calculation is motivated by the standard
%% one of federal research overhead rates.  In this context, UCB faculty
%% does very well bringing in roughly \$748 million in research funds in
%% 2015, and the overhead on this of \$175 million.  Indeed they bring
%% millions more on graduate student researcher tuition goes directly to
%% the university for roughly 2000 GSR positions.  Faculty (non-summer,
%% non-lecturer) salary in our data is around \$210 million.  We do
%% categorize that as directly serving the mission.


We note that while we focus exclusively on employee compensation in our analysis, 
there is perhaps something to learn from best practices
at other UC's in categories other than employee compensation. For example, it appears UCSB 
escapes significant expenses (on the order of \$77 million) in student
healthcare which Berkeley does not (and to be fair, neither does
UCLA); see section~\ref{sec:healthcare}. 


This is a working document, and we welcome feedback about any gaps in our
analysis.


\section{Overhead Calculations}

We proceed in the next sections with calculations that use
two data sources with different views of salary expenditures.
The two yield similar results.  

\subsection{IPEDS 2014.}

The Integrated Postsecondary Education Data System (IPEDS) is
maintained by the National Center for Education System and collects
information from colleges and university using standardized
methodologies.

We use these data to compare UC Berkeley and UC Santa Barbara effective overhead
rates overall in this section, and on specific functions in the next
subsection.

For instructional salary, we have UC Berkeley at \$232 million
compared to UC Santa Barbara's \$112 million from the IPEDS 2014
datafiles at \cite{ipeds-is}.  For non instructional salary totals, we
have UCB at \$653 million and UCSB at \$208 million from the IPEDS
2014 datafile at \cite{ipeds-nis}.

The non-instructional salaries include research salaries which are
\$82 million (UCB) and \$25 million (UCSB) which should not be
considered overhead. Moreover, UCB spends significantly more than UCSB, \$25
million versus \$4 million, on the category of ``Librarians, Curators,
Archivists and Academic Affairs and Other Education Services -
outlays'' which we don't feel comfortable categorizing as overhead.

Deducting those costs from the total non-instructional salary reported
to IPEDS leaves \$546 million in overhead salaries for UC Berkeley
versus \$179 million for UC Santa Barbara.

{\bf Basic Calculation.} Putting this together, the effective overhead
rate for UCB is 235\%, the ratio of \$546 million to \$232
million. The effective overhead rate for UCSB is 159\%, the ratio of
\$179 million to \$112 million.  Were Berkeley consistent with Santa
Barbara on this overhead, its non-instructional salaries would
engender a savings of \$177 million in salary alone.

Using an estimated associated benefit rate of 36\% as suggested by
\cite{rev-trends} yields a saving of \$240 million dollars in salaries
and benefits alone based on 2014 data.  

\iffalse
%%%The following three paragraphs are confusing. Move to a separate subsection???
Of course, one could argue that research induces overhead as well.
%% The first yields a possibly illegal overhead rate on federal funds of
%% 174\% for Berkeley.  
Thus, we include a calculation where the rate is
limited to 57\%: the federally negotiated overhead on research grants.
\fi

%% {\bf Large Research Overhead Calculation.} We add research and faculty
%% salaries which yield totals \$314 million (UCB) and \$137 million
%% (UCSB). The associated overhead rates are then respectively 174\% and
%% 152\%.  The associated savings for Berkeley were these equal wouldbe  94
%% million (using the calculation $1.36\times (546 - 314\times1.52)$).

{\bf Deducting Research Overhead Calculation.} Federal research has a
fixed overhead rate, which is 57\%.  Thus, we remove 57\% of \$82
million from the \$546 million of overhead from our calculation to get
overhead salary spending of 499 million for UCB and an analogous
calculation yields \$165 million for UCSB.  We then compute an effective
overhead rate of 215\% and 147\% respectively for UCB and UCSB.
Reducing Berkeley's overhead to that of UCSB would save \$214 million
on total compensation.\footnote{The computation is $(546 - .57*82)/232 = 2.15$
for Berkeley and $(179 - .57*25)/112 = 1.47$ for Santa Barbara. It begins
with the overhead salary and removes overhead allocated to research
according to federal formula and then computes the effective overhead
rate. This computation presumes neither spends more than the federally
mandated overhead rate on research.}


\subsubsection{Specific areas for further examination.}

These are from the IPEDS non-instructional staff data
\cite{ipeds-nis}.  

In the following, we are reporting UCB versus UCSB numbers for various
functions.  We compute savings by reducing the per function overhead
rate of UCB to that computed for UCSB and reporting the difference in
salary plus estimated total compensation.  We calculate overhead rates
by dividing salaries for the specific function by teaching
salaries which are \$232  million for UC Berkeley and \$112 million for
UC Santa Barbara.\footnote{We do this for consistency with the other
  numbers, in some sense the sum of the overheads reported in all the
  functional areas of which some are listed should equal the total
  reported in the first calculation above.}  We then calculate compensation savings
by computing the difference in dollar amount of salary
between UCB actual salary and what UCB would spend if its
overhead rate were the same as UCSB.  Dollars are reported
with an M, indicating millions.

\begin{tabular}{|l | c | c | c | c| c|}
\hline
Function & UCB Salary & UCB rate & UCSB Salary & UCSB rate & Savings \\
\hline
{\bf Management} & \$136M & 59\% & \$29M & 26\% & \$102M \\
{\bf Business/Finance.} & \$97M  & 42\% & \$31M  &28\% & \$43M  \\
{\bf Office/Admin Support.} & \$69M & 30\% &
\$20M & 18\% & \$38M \\
{\bf Student Services.} & \$64M  & 28\% & \$19M  &
 17\% & \$34M\\
{\bf Computer,etc.} & \$105M  & 45\% & \$37M  & 33\%&  \$38M \\
\hline
\end{tabular}


\subsection{Salary data}

We also worked with salary data files provided by the California State
Controller which includes per employee title along with `Regular Pay'
and `Total Wages'\cite{salary}.  This data is a different way of
looking at the picture and includes all employees.  Our analysis is
heuristic but yields conclusions consistent with those above.

Here we categorized titles as direct teaching/research and public
service titles as follows: we select all employees who are any form of
Professor, Lecturer, Teacher, or Teaching Assistant and total their
salaries for the teaching category, and select all Agronomists, Museum
titles, and Curators, Research Scientist and Graduate Student
Researchers and total their salaries for the research
category.\footnote{This may well include titles associated with, for
  example, Lawrence Hall of Science, but we conservatively included
  these anyway.} We pulled and scanned the full list of job titles to
include any that might be in this category. We define direct salary
for employees with these titles, and define the salary of the
remaining employees as overhead salary. Note that here we include
both teaching and research in the direct salaries but the data
also includes others that do not appear in IPEDS data.

For full details,  code at \cite{github-link} embodies our selection heuristics. 

{\bf Regular Pay.}  Using the {\bf `Regular Pay'} field for the direct
employees described above yields \$379 million out of \$1082 million
total salary and \$173 million out of \$415 million total salary for
UCB and UCSB respectively. We conclude overheads are \$703 million
and \$242 million for UCB and UCSB respectively.

This yields effective overhead rates of 185\% and 140\% for UCB and
UCSB respectively.  Again, if UCB overhead rate was reduced to that of
UCSB, Berkeley would save Berkeley \$170 million in salary or \$232
million in total compensation.

{\bf Total Wages.} Using {\bf `Total Wages'} rather than base pay
(e.g., including summer salary), we get \$440 million out of \$1185
million and and \$194 million out of \$447 million for UCB and UCSB
respectively.  We conclude overheads are \$745 and \$253 for UCB and UCSB
respectively. 

This yields effective overhead rates of 169\% and 130\%.  Equalizing
would save Berkeley \$171 million in salary or \$233 million in total
compensation. 

{\bf Adjusting for auxiliary.} One could legitimately remove things
like housing associated salaries from overhead since they are paid for
from revenues generated by those activities themselves, even
understanding that room and board fees may well be used for a wide
variety of purposes.  Thus, we use data from 2015 Consolidated
Financial Reports for UCB and UCSB
\cite{UCSB-consolidated,UCB-consolidated} which suggest that auxiliary
salaries and wages are \$46 million and \$36 million respectively.

With these adjustments to the total wage scenario we get effective
overhead rates of 158\% and 112\%.  Bringing Berkeley's overheads to
UCSB's level would save Berkeley \$202 million in salary or \$274
million in total compensation.



\section{Other Questions/Issues.}
\label{sec:healthcare}

\begin{enumerate}
\item
{\bf Student health expenditures} from the Berkeley Consolidated
Report \cite{UCB-consolidated}  2015 has a non-salary expense of
\$72 million.  For Santa Barbara \cite{UCSB-consolidated} this is
\$11 million. Admittedly, UCLA  \cite{UCLA-consolidated}
also reports large non-salary expense for this category, but it
is curious that Santa Barbara escapes it.

\item
{\bf The number of Affiliates/Non-Employees} has moved from 1921 in
2008 to 4209 in 2016 \cite{cal-answer-census}.  Who are these people?
Do they incur expense? Are their salary expenses included in either
the data submitted to the Federal government via IPEDS, or to the
State Controller?

\item
{\bf Other Provisions} increased from a \$20 million in the 2014 Consolidated
Financial Statement \cite{UCB-consolidated-14} to \$123 million in the 2015
Consolidated Financial Statement \cite{UCB-consolidated}.  This amount
is assigned to Cost of Instruction (as for example is the vast
majority of the Deans offices).  While we received some information
that these are uncategorized expenses at the time of report preparation,
it does seem to contribute to a roughly comparable increase in the
total cost of instruction year over year as well between these two statements.

\item
{\bf Financial Aid}. We can also examine financial aid in terms or
spending again by continuing to use instructional faculty as the
denominator.  Thus, the \$135 million UCB spends results in a 58\%
effective overhead rate where \$83 million that UCSB spends results in
a 74\% overhead rate.  Here Berkeley spends significantly less than
Santa Barbara in this normalized sense.  This perhaps due to demographic
differences but does help Berkeley with respect to expenses making
its budget deficit a bit more puzzling. 


\end{enumerate}

%% \section{Some remarks.}

%% For Berkeley to function, one certainly needs many people beyond
%% teachers and researchers to properly accomplish our mission. For
%% example, we certainly need an admissions office and a registrar. These
%% absolutely central tasks, however, cost a modest \$4.5 million and \$3.5 million in total.
%% Berkeley, moreover, is slightly more efficient than Santa Barbara in this
%% categories.

%% For comparison purposes, Dean's offices total expenses are around
%% \$47.5 million and Vice Chancellors another \$34 million.  Again, Santa Barbara does
%% better in these categories.

%% Incidentally, faculty brought in roughly \$748 million in research funds in
%% 2015, and the overhead on this of \$175 million.   Moreover, millions more on
%% graduate student researcher tuition goes directly to the university
%% for roughly 2000 GSR positions.  For comparison purposes faculty
%% (non-summer, non-lecturer) wages are around \$210 million.




\bibliographystyle{plain}
\bibliography{reports}


\end{document}
