\documentclass{article}

\input{macros}
\input{macros_std}
\usepackage[utf8]{inputenc}
\usepackage[T1]{fontenc}
\usepackage{lmodern} 


\title{}
\author{}

\newcommand{\pt}{{\cal{P}_{\cal{T}}}}
\newcommand{\ptnull}{{\cal{P}_{\cal{T_{\perp}}}}}
\newcommand{\sampleq}{{\tilde{\cal R^q}}_{\Omega}}
\newcommand{\sampleqk}{\tilde{{\cal R^q}}_{\Omega_k}}
\newcommand{\samplesk}{{\cal R}_{\Omega_l}}
\newcommand{\samples}{{\cal R}_{\Omega}}

\newcommand{\xxx}[1]{\newline {\bf XXXXXX:} {\em #1} \\}


\documentstyle{article}

\begin{document}

We present a discussion of Berkeley's use of its funds largely by
comparison with UC Santa Barbara. The ongoing analysis centers on
salaries which along with associated benefits comprise the bulk of
Berkeley's expenditures.  It appears that Berkeley could save in the
neighborhood a low estimate of around 100 million to estimates which
suggest more than 200 million dollars were it consistent with UC Santa
Barbara's use of funds with respect to salaries beyond direct
teaching, research, and public service missions.

We make no claim nor do we believe UC Santa Barbara especially
effective, just more effective than Berkeley. 

We note this is an ongoing process and include questions to perhaps
the administration which may add clarity or correct misinformation
and welcome all other helpful feedback especially but not limited
to pointing calculation errors which we must surely have made.

We note there is perhaps something to learn from best practices
elsewhere beyond employee compensation. For example, it appears
Santa Barbara escapes huge expenses in student healthcare
which Berkeley and to be fair, Los Angeles, do not. See
section~\ref{sec:healthcare}.

\section{Overhead Calculations}

\section{IPEDS 2014.}

From 2014 ipeds datafiles \cite{ipeds-is}, we have that UC
Berkeley spent 232 million dollars on faculty salary to UC Santa
Barbara's 112 million.  For non instructional salary totals,
we have 653 million versus 208 million \cite{ipeds-nis}. 

The non-instructional salaries include research salaries which are
respectively 82 million and 25 million, and Berkeley also spends
significantly more, 25 million versus 4 million, on the category of
``Librarians, Curators, Archivists and Academic Affairs and Other
Education Services - outlays'' which we also take to directly serve
our mission.  

Deducting those costs leaves 546 million versus 179 million for 
UC Santa Barbara. 

{\bf Basic Calculation.} The effective overhead of this calculation for UC Berkeley is 235\%,
and 159\% for UC Santa Barbara.  Were Berkeley consistent with Santa
Barbara on this overhead, its noninstructional salaries would engender
a savings of 173 million in salary alone: 546 million versus imputing
$(1.59 \times 232)$ million. 

Using and estimate an associated benefit rate of 36\% using Berkeley
overhead numbers indicated in \cite{rev-trends}, This yields a saving
235 million dollars in salaries and benefits alone using data from
2014 data.  More recent data which we discuss below suggests a
widening gap between Berkeley and Santa Barbara.

Alternatively, one could argue that research induces overhead
as well.  To be sure, it should be on the order of 57\%. Thus,
we adjust this calculation in two ways. 

{\bf Large Research Overhead Calculation.} We add research and faculty
salaries which yield 314 million and 137 million respectively. The
associated overhead rates are then respectively 174\% and 152\%.  The
associated savings for Berkeley were these equal 94 million (using the
calculation $1.36\times (546 - 314\times1.52)$).

{\bf Modest Research Overhead Calculation.} On the other hand, were
the research overheads limited to say 57\% (our typical research
overhead rate), we remove that from the overheads instead getting
``teaching overheads'' of 499 million ($546 - .57\times 82$) and 165
million, respectively.  This teaching overhead rate then becomes 215\%
and 147\%.  Equalizing that overhead would save 190 million.

\subsubsection{Some places to look.}

Management

Finance

Business

Student Services

Sudent Health payment. 

\subsection{Salary data}

We also worked with salary data files \cite{salary,ucpay}.

Here we heuristically computed direct and indirect salary expenditures
for 2015. The computation is embodied in a publicly available
repository \cite{github-link}. 

For that year, we estimate total direct salaries (teaching,research,
museum, agronomists) as 360 million and 164 million out
of total salaries 1055 million and 394 million for Berkeley
and Santa Barbara respectively.  This is using the regular
pay fields in that data. 

Here, the imputed salary overhead rates are 194\% and 140\%. 
Equalizing yields 


\bibliographystyle{plain}
\bibliography{reports}


\end{document}
